% \newcommand{\prototitle}{Versuch 2 - Statistik}
% \newcommand{\Fachbereich}{Praktikum Messtechnik}
% \input{../packages/tu_header}

\newcommand{\institut}{Institut f\"ur Telekommunikationssysteme}
\newcommand{\fachgebiet}{Nachrichten\"ubertragung}
\newcommand{\veranstaltung}{Praktikum Nachrichten\"ubertragung}
\newcommand{\pdfautor}{\"Ozg\"u Dogan (326 048), Boris Henckell (325 779)}
\newcommand{\autor}{\"Ozg\"u Dogan (326 048)\\ Boris Henckell (325 779)}
\newcommand{\gruppe}{Gruppe: }
%\newcommand{\betreuer}{Betreuer: Mahmoud Felk}


\newcommand{\pdftitle}{Nachrichten\"ubertragung\ Praktikum\ 03}
\newcommand{\prototitle}{Praktikum 03 \\ Einf\"uhrung in MATLAB}


\input{../../packages/tu_header_8}


% \lstlistoflistings
\definecolor{darkgray}{rgb}{0.95,0.95,0.95}
\definecolor{darkolivegreen}{HTML}{01a801}
\definecolor{functionsBlue}{HTML}{32b9b9}
\definecolor{variableRed}{rgb}{1,0,0}
\definecolor{stringBrown}{HTML}{bc8e8e} % f geht nicht

\lstset{
        %\lstset{extendedchars=true} % Umlaute an der richtigen stelle und nicht am Anfang ausgeben
        %basicstyle=\footnotesize\ttfamily,
        basicstyle=\small,
        %
        inputencoding=utf8,
        %
        tabsize=4,
        showspaces=false,
        showtabs=false,
        showstringspaces=true, % no special string spaces
        %
        backgroundcolor=\color{darkgray}, % background
        stringstyle=\color{stringBrown}\fseries, % Strings
        keywordstyle=\color{functionsBlue}\bfseries, % keywords Blau
        identifierstyle=\color{variableRed}, % variablen
        commentstyle=\color{darkolivegreen}, %  comments
        %
        breaklines=true,
        %
        numbers=left,
        numberstyle=\tiny,
        stepnumber=1,
        numbersep=7pt,
        %
        frame=single,
        columns=flexible,
        %
        xleftmargin=-2cm,
        xrightmargin=-1.5cm,
        %
        language=Matlab
}

%---------------------------------------------------------------------
%---------------------------------------------------------------------
%---------------------------------------------------------------------

\section{Einleitung}
\begin{quote}
	In diesem Termin wurde durch praktisches Aufbauen und Testen von modulierenden
	Übertragungsstrecken das Prinzip der Amplitudenmodulation (AM) und der
	Frequenzmodulation (FM) nachvollzogen. Dafür wurde zuerst immer das nötige
	Signal erzeugt, welches dann moduliert und auch demoduliert wurde.
\end{quote}

\section{Amplitudenmodulation}
\begin{quote}
	\subsection{Theorie}
    \begin{quote}
        
    \end{quote}
    
    \subsection{Vorbereitungsaufgabe}
    \begin{quote}
     
    \end{quote}
    
    \subsection{Durchführung}
    \begin{quote}
                    Als Basisbandsignal wurde ein mittelwertfreier Sinus erzeugt, welcher die
            Frequenz
    \end{quote}
    
    \subsection{Auswertung}
    \begin{quote}
        
    \end{quote}
\end{quote}

%--------------------------------------------------------------------
%--------------------------------------------------------------------            


\section{Frequenzmodulation}
\begin{quote}
    \subsection{Theorie}
    \begin{quote}
        
    \end{quote}
    
    \subsection{Vorbereitungsaufgabe}
    \begin{quote}
     
    \end{quote}
    
    \subsection{Durchführung}
    \begin{quote}
        
    \end{quote}
    
    \subsection{Auswertung}
    \begin{quote}
        
    \end{quote}	
\end{quote}



\end{document}
