% \newcommand{\prototitle}{Versuch 2 - Statistik}
% \newcommand{\Fachbereich}{Praktikum Messtechnik}
% \input{../packages/tu_header}

\newcommand{\institut}{Institut f\"ur Telekommunikationssysteme}
\newcommand{\fachgebiet}{Nachrichten\"ubertragung}
\newcommand{\veranstaltung}{Praktikum Nachrichten\"ubertragung}
\newcommand{\pdfautor}{\"Ozg\"u Dogan (326 048), Boris Henckell (325 779)}
\newcommand{\autor}{\"Ozg\"u Dogan (326 048)\\ Boris Henckell (325 779)}
\newcommand{\gruppe}{Gruppe: }
%\newcommand{\betreuer}{Betreuer: Mahmoud Felk}


\newcommand{\pdftitle}{Nachrichten\"ubertragung\ Praktikum\ 03}
\newcommand{\prototitle}{Praktikum 01 \\ Einf\"uhrung in MATLAB}


\input{../../packages/tu_header_8}


% \lstlistoflistings
\definecolor{darkgray}{rgb}{0.95,0.95,0.95}
\definecolor{darkolivegreen}{HTML}{01a801}
\definecolor{functionsBlue}{HTML}{32b9b9}
\definecolor{variableRed}{rgb}{1,0,0}
\definecolor{stringBrown}{HTML}{bc8e8e} % f geht nicht

\lstset{
        %\lstset{extendedchars=true} % Umlaute an der richtigen stelle und nicht am Anfang ausgeben
        %basicstyle=\footnotesize\ttfamily,
        basicstyle=\small,
        %
        inputencoding=utf8,
        %
        tabsize=4,
        showspaces=false,
        showtabs=false,
        showstringspaces=true, % no special string spaces
        %
        backgroundcolor=\color{darkgray}, % background
        stringstyle=\color{stringBrown}\fseries, % Strings
        keywordstyle=\color{functionsBlue}\bfseries, % keywords Blau
        identifierstyle=\color{variableRed}, % variablen
        commentstyle=\color{darkolivegreen}, %  comments
        %
        breaklines=true,
        %
        numbers=left,
        numberstyle=\tiny,
        stepnumber=1,
        numbersep=7pt,
        %
        frame=single,
        columns=flexible,
        %
        xleftmargin=-2cm,
        xrightmargin=-1.5cm,
        %
        language=Matlab
}

%---------------------------------------------------------------------
%---------------------------------------------------------------------
%---------------------------------------------------------------------
\section{Einleitung}




\section{Vorbereitungsaufgaben}

    \begin{quote}
    Die Vorbereitungsaufgabe lautet, die Varianz von gleichverteiltem weißen Rauschen $N$ mit der
    Verteilungsdichtefunktion $p_{N(n)} = \frac{1}{2A} \sqcap_{2A} (n)$ zu berechnen.\\
    Zuerst berechnen wir den Mittelwert $\mu$:
   	\begin{equation*}
    	\begin{split}
    		\mu &= \int_{-\infty}^{+\infty} n \frac{1}{2A} \sqcap_{2A} (n) \mathrm dn\\
    		&= \int_{-A}^{+A} n \frac{1}{2A} \mathrm dn\\
    		&= \left[ \frac{1}{2A} \frac{1}{2} n^2 \right]_{-A}^{+A}\\
    		&= \frac{1}{4A} (A^2-(-A)^2)\\
    		&= 0
    	\end{split}
    \end{equation*}
    
    Dannach berechnen wir die Leistung des Signals$P$:\\
    \begin{equation*}
    	\begin{split}
    		P &= \int_{-\infty}^{+\infty} n^2 \frac{1}{2A} \sqcap_{2A} (n) \mathrm dn\\
    		&= \int_{-A}^{+A} n^2 \frac{1}{2A} \mathrm dn\\
    		&= \left[ \frac{1}{2A} \frac{1}{3} n^3 \right]_{-A}^{+A}\\
    		&= \frac{1}{6A} (A^3-(-A)^3)\\
            &= \frac{2A^3}{6A} = \frac{1}{3} A^2\\
    	\end{split}
    \end{equation*}
    
    Mit diesen Werten läßt sich die Varianz wie folgt berechnen:
    \begin{equation*}
    	\begin{split}
    		\sigma^2 &= P - \mu^2\\
    		&= \frac{1}{3} A^2 - 0 = \frac{1}{3} A^2
    	\end{split}
    \end{equation*}
	\end{quote}
         	


%--------------------------------------------------------------------
%--------------------------------------------------------------------            
\section{Praktische Aufgaben}
\begin{quote}
   
\end{quote}	




%--------------------------------------------------------------------
%--------------------------------------------------------------------
\section{Matlab-Code}
\begin{quote}
%     \subsection{Aufgabe1.m bearbeitet}
%     \begin{quote}
%             \lstinputlisting[
%             caption={Aufgabe1 - Matlab-script},
%             label=lst:Matlab]
%             {./Matlab/Aufgabe1.m}
%     \end{quote}
%     
%     \subsection{Aufgabe2.m bearbeitet}
%     \begin{quote}
%             \lstinputlisting[
%             caption={Aufgabe2 - Matlab-script},
%             label=lst:Matlab]
%             {./Matlab/Aufgabe2.m}
%     \end{quote}
%     
%     \subsection{SNR.m bearbeitet}
%     \begin{quote}
%             \lstinputlisting[
%             caption={SNR - Matlab-script},
%             label=lst:Matlab]
%             {./Matlab/SNR.m}
%     \end{quote}
    	
\end{quote}


\end{document}
